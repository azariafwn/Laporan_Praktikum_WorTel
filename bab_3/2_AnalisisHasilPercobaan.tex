% Bagian Analisis Hasil Percobaan
\section*{Analisis Hasil Percobaan}
\indent
Pada Percobaan modul 1 WORKSHOP Telematika, workshop dimulai dengan membuat rangkaian schematics minimum system ESP8266 yang dilakukan dengan mengikuti langkah-langkah yang sudah dicantumkan di modul. Proses membuat rangkaian schematic ini menggunakan komponen-komponen yang terdapat pada library yang sudah disediakan oleh aslab. Setelah proses membuar rangkaian schematic selesai, kemudian dibuat PCB dari schematic tersebut. Setelah proses wiring koneksi selesai kita mendapatkan hasil PCB yang siap di gunakan.

% \begin{table}[h]
%     \centering
%     \caption{Caption tabelnya}
%     \label{tab:labelini}
%     \begin{tabular}{|c|c|c|c|}
%     \hline
%     Kolom 1 & Kolom 2 & Kolom 3 & Kolom 4 \\
%     \hline
%     Data 1 & Data 2 & Data 3 & coba nambah kolom \\
%     Data 4 & Data 5 & Data 6 & coba nambah kolom juga \\ 
%     \hline
%     \end{tabular}
% \end{table}