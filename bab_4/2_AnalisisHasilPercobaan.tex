% Bagian Analisis Hasil Percobaan
\section*{Analisis Hasil Percobaan}
\indent
Pada bab 4 ini dipelajari terkait ESP8266 Minimum System. Pada saat praktikum, praktikan diarahkan untuk memasang secara mandiri setiap komponen ke board pcb yang telah disediakan. Peletakan komponen dan nama-nama komponen telah tersedia lengkap pada modul github yang disediakan aslab. 
\\ \indent 
Setelah semua pemasangan komponen selesai, dilakukan pemanasan terhadap PCB tadi diatas alat pemanas. Fungsi alat ini menggantikan solder uap. Praktikan tidak dianjurkan menggunakan solder uap dikarenakan jika di solder uap dari bagian atas maka komponen nya akan banyak yang gosong karena tidak tahan panas jika terpapar langsung dari sisi atas. Maka dari itu dilakukan pemanasan dari bagian bawah PCB dengan alat pemanas.
\\ \indent 
Setelah proses pemanasan selesai dan semua komponen sudah menyatu pada tempatnya dengan baik, lalu dilakukan proses soldering dengan solder biasa pada komponen komponen tersisa yang belum terpasang.
\\ \indent 
Setelah semua proses soldering selesai, dilakukan proses upload firmware (yang berisi kode yang telah dikerjakan saat tugas pendahuluan) ke ESP menggunakan PlatformIO. Setelah program berhasil diupload lalu dilakukan percobaan, apakah antara kode yang diupload dan perilaku komponen (LED dan Button) telah sesuai. Saat perilaku sudah sesuai maka dilakukan dokumentasi dan praktikum selesai.
