% Bagian Analisis Hasil Percobaan
\section*{Analisis Hasil Percobaan}
\indent
Pada modul 2 ini dilakukan kegiatan menyolder rangkaian pada pcb dengan menggunakan solder biasa dan solder uap. Pada percobaan pertama, yaitu soldering dengan solder biasa pada pcb, praktikan menyolder komponen white housing (namun bukan menggunakan komponen white housing nya melainkan menggunakan komponen penggantinya), komponen resistor, serta komponen 7 segment display. Praktikan mempelajari teknik soldering yang tepat dan mempelajari bagaimana 7 segment display bekerja sesuai pin yang terhubung ke vcc. 
Pada percobaan kedua, dilakukan proses soldering menggunakan solder uap. Praktikan menyolder komponen SMD menggunakan solder uap dan solder paste mengikuti langkah-langkah yang telah terdapat di modul serta dibantu oleh asisten lab. Komponen yang disolder menggunakan solder uap adalah kapasitor, resistor, serta ic. Lalu setelah selesai menyolder komponen SMD dengan solder uap dilanjutkan dengan menyolder komponen yang membutuhkan proses soldering menggunakan solder biasa. 




% \begin{table}[h]
%     \centering
%     \caption{Caption tabelnya}
%     \label{tab:labelini}
%     \begin{tabular}{|c|c|c|c|}
%     \hline
%     Kolom 1 & Kolom 2 & Kolom 3 & Kolom 4 \\
%     \hline
%     Data 1 & Data 2 & Data 3 & coba nambah kolom \\
%     Data 4 & Data 5 & Data 6 & coba nambah kolom juga \\ 
%     \hline
%     \end{tabular}
% \end{table}